\documentclass[preprint]{aastex}
\synctex=1
\usepackage{graphicx}
\usepackage{amssymb,amsmath}
\usepackage{natbib}
\shorttitle{Astro 540 Term Paper}
\shortauthors{Yifan Zhou}
\slugcomment{Astro 540 Term Paper}

\begin{document}
\bibliographystyle{abbrvnat}
\title{The Mass of the Milky Way}
\author{Yifan Zhou}
\affil{University of Arizona}
\email{yifzhou@email.arizona.edu}

\begin{abstract}
Mass is one of the most essential properties of the Milky Way. Mass is
measured with the combination of observed dynamics and theoretical
structures. In paper, I introduced the major techniques to measure the
rotation curve of the Milky Way and other observational constrains of
the mass of the Milky Way. I gave a brief overview of the mass models
of the galaxy. And finally I discussed about the limitation of the
measurement and potential improvements.  
\end{abstract}

\section{Introduction}

Mass is one of the most fundamental properties of the Milky Way. Since
the Milky Way is a system that is bond by gravity, its mass is a major
driver of the Milky Way's dynamic evolution. On the other hand, the
mass distribution of the Milky Way provided us a most direct view of
the structure of the Milky Way. Also, comparing the mass measured from
kinematic with the matter that we can see in the Milky Way, we can
obtain the mass to light ratio in the Milky Way, which tells us
whether there is dark matter in the Milky Way.

The essentiality of mass in our understanding of the Milky Way calls
for a precise measurement. In most of the cases, to measure the mass
of the Milky Way is to measure the gravity it generates. Thus the most
straight forward scale of mass is the rotation curve, because in the
first order, the gravity's role is control the matter to orbit in the
galaxy. Many techniques are developed to measure the rotation curve at
different galactocentric radii. Except for rotation curves, more
complicated effects of gravity such as tidal effect could also provide
us with observational constrains of the mass.

However, dynamic measurement itself cannot provide us with the
mass. Galaxies with different mass could possibly provide the same
rotation curve if they have different mass distribution. Therefore, we
need mass models. Mass models represent our abstract recognition of
the galaxy, describe the construction of the Milky Way and the
profiles of different component. To connect theoretical model with
observational measurement, we can define the free parameters in the
models, including the most important one, mass.

\section{Rotation Curves}

The principle of mass measurement with
rotation curves is simple.

\begin{equation}
  \label{eq:vc} vc^{2}=-\nabla_{r}\Phi
\end{equation}

Equation \ref{eq:vc} is also easy to be decomposed with
different components. The geometry of Rotation curve measurement is
also as straight forward. As shown in figure \ref{fig:tri}, if the
distance $r$ and radial velocity $v_{r}$of a specific object with
galactic longitude $l$ are known, and if we assume that we know the
position ($R_{0}$)and the motion ($v_{0}$)of sun in the Milky Way, we
can easily calculate the velocity $v(R)$ and orbit radius $R$ of it
with equation \ref{eq:tri} \cite{2013pss5.book..985S}. However the measurement of
distance and velocity could be difficult.

\begin{equation}
  \label{eq:tri}
  \begin{split} &R = \sqrt{r^{2} + R_{0}^{2} - 2rR_{0}\cos l}\\ &v(R)
= \frac{R}{R_{0}}\left(\frac{v_{r}}{\sin l}+v_{0}\right)
  \end{split}
\end{equation}

\begin{figure}[!t] \centering \plotone{tri}
  \caption{Rotation curve measurement and tangential point diagram
\cite{2013pss5.book..985S}}
  \label{fig:tri}
\end{figure}

\subsection{Rotation curve inside of the solar circle}

  For the region that is inside of the solar circle, tangential point
provide us with a shortcut of rotation curve measurement. For this
region, the rotation curve is often measured with H I 21 cm line or CO
lines. If we observe H I 21 cm line with a specific galactic longitude
$l$, the maximum radial velocity $v_{r,\rm max}$will fall on the
tangential point as shown in the right panel of figure
\ref{fig:tri}. Thus equation \ref{eq:tri} can be simplified as,
  \begin{equation}
    \label{eq:tan}
    \begin{split} &v(R)=v_{r, \rm max}+v_{0}\sin l\\ &R=R_{0}\sin l
      \end{split}
    \end{equation}
    which means we do not need to worry about distance
measurement.\par

    As an example, \citet{1994ApJ...433..687M, 1995ApJ...448..138M} used H I 21 cm
line to measure the rotation curves. In this measurement, the velocity
dispersion is smaller than 9 km s$^{-1}$.
    % \begin{figure} % \centering % \plotone{inner}
    % \caption{Terminal velocity measured inside of the solar %
circle. $V_{\rm T} = v(R)-v_{0}\sin l$ % cite{}}
    % \label{fig:inner}
    % \end{figure}
%
    \subsection{Rotation curve outside of solar circle}
    
    The tangential point disappears in the outer region of the Milky
Way. Therefore the distance measurement becomes important in this
circumstances. \par

   \citet{1993A&A...275...67B} used H II region and reflection nebulae to
measure the rotation curves. The distance of the region was measured
by photometric and spectroscopic fitting. The expand the boundary of
rotation curves to around $2R_{0}$.\par

    \citet{1997A&A...318..416P} gived the radial velocities and photometry for
Cepheids in the outer disk. For Cepheids stars, the distance could be
calculated with the period-luminosity relation. \par

    There were other methods being proposed to measure the rotation
curves in outer disk, e.g. using the H I 21 cm lines. However, there
are still large uncertainties and biases for the distance measurement.
%

\subsection{Result}

Combine several measurements together, a
rotation curve within $2R_{0}$ could be obtained (figure
\ref{fig:rc}). The most prominent feature on the rotation curve is its
flatness in large radii, which is contradict with the disk model of
the galaxy.

     \begin{figure}[!t] \centering \plotone{rc}
       \caption{Observed circular velocities representing the rotation
curve of the galaxy \cite{2013pss5.book..985S}.}
       \label{fig:rc}
     \end{figure}
     
     \section{Other Constrains}
          
     Other observational constrains are brought in to improve the
accuracy of the milky way mass measurement.
     
\subsection{Local constrains}

For all the rotation curve measurements listed above, the rotation
velocities and orbit radii are all based on sun's motion. However,
solar motion and position are not well constrained as expected. For
example, \citet{2009ApJ...700..137R} reported VLBI measurements of
trigonometric parallaxes and proper motions for 18 masers located in
several of the Galaxy’s spiral arms. These measurements yielded the
circular rotation speed at the position of the Sun for
$v_{r,0} = 254 \pm 16 \rm{km s−1}$, which is 15\% higher than the
standard IAU 220kms$^{−1}$. Several other measurement did not reduce
the scatter of the measurement.

     Local surface density also provide constrain on model
fitting. However, this value is also not well measured \cite{1998MNRAS.298..387D}.

     \citet{2007MNRAS.379..755S} dug local information in another point of
view. They used a sample of high-velocity stars from the RAdial
Velocity Experiment to calculate the local escape speed.  Their result
demonstrates that the local escape speed is significantly higher than
$\sqrt{2}v_{c}$, which gives a hint of the dark matter halo.

\subsection{Kinematic tracer at large radii}

The rotation curves are roughly limited at the radius of 2 times of
the solar orbit radius. However, observations indicate that this
radius is still far from the edge of the galaxy, especially for the
mysterious dark matter halo. Thus mass measurement at large radii is
urgently needed.

     \citet{2008ApJ...684.1143X} used Blue Horizontal Branch (BHB) stars to push the
edge of the galaxy mass measurement to 60 kpc. BHB stars are not only
very bright, but have small scatter in luminosity. Therefore they
would provide a very good photometric distance measurement at large
distance. With BHB stars observation, they derived a rotation curve
by comparing their data with mock observations of simulated galaxies
at such large radius.

     As satellite galaxies gradually becomes well-known, they are also
used as a mass tracer. \citet{2010MNRAS.406..264W} took a sample consisting of 26
satellite galaxies with line-of-sight velocities to estimate the mass
of the Milky Way within a radius of 300 kpc.
     
\section{Mass Models}
Mass models are the prior assumption of the
mass distribution of the Milky Way. Generally, mass models always have
three components, the bulge, disk and dark matter halo. Mass Modeling
is possible because the gas particles and stars are sensitive to the
full potential contributed by both baryon and dark matter. If we
decompose the total circular velocity,
     \begin{equation} v_{c} \sim \sqrt{v_{\rm gas}^{2}+v_{\rm
star}^{2} + v_{\rm halo}^{2}}
     \end{equation} Gas and star component could come from the
observation of luminous matter. Halo components on the other hand
comes from the total circular velocity subtracted the star and halo
part. Each component has different mass distribution profile, ending
up with different format of gravity potential.

     Bulge is studied with the observation in the galactic
center. Proper motion studies in the near infrared have revealed
individual orbits of stars within the central 0.1 pc
\cite{2013pss5.book..985S}. Outside the very nuclear region, radial velocity of
OH and SiO maser lines from IR stars in the galactic center region are
used to study the kinematics. From these study, the bulge is revealed
to have a near-prolate, triaxial rotating profile. However in the
modeling, axisymmetric approximation is often
applied \cite{2011MNRAS.414.2446M}.

     From the observation luminous matter, e.g. star number density,
the Milky Way's disk is usually considered to have two components, the
thin disk and the thick disk. The two disk are always modeled with
exponential profile with different scale height and scale length.

     The flatness of rotation curve at large radii implies there is
huge mass in the Milky Way that we don't see. These mass is modeled
with dark matter halo. On observational side, the only trace of dark
matter is the dynamics at large radii. On theoretical side,
simulations provide the profile of the dark matter halo. In the mass
Model, NFW profile is often used.

     In the model, several parameters related to mass, including local
disk surface density, local dark matter density, bulge density. These
parameter are fixed by fitting the observation rotation curve to the
model. Then the mass could be calculated.
     
     \section{Discussion} In recent studies, the results of mass
measurement fall in the range around $10^{12} M_{\odot}$. However, the
scatter is considerable. Several improvement could be made for the
precision of the mass measurement.

     The rotation curve at outer disk could be the first
consideration. GAIA is starting a new era of parallax measurement, and
is going to provide with more accurate distance at larger radius.

     The mass estimators on the edge of the galaxy need to be further
studied, Since they are the best constrain of the dark matter
halo. Except for the contribution in mass measurement, these study
could disclose more information of the property of the halo and dark
matter itself.

     The limitations of the mass models also need to be overcome. The
most thorny issue of mass modeling is the intrinsic degeneracy of the
model solution due to strong covariances between the disk and halo
model parameters. To break the degeneracy, we need more observation
constrains.
\bibliography{ref.bib}
\end{document}

%%% Local Variables:
%%% mode: latex
%%% TeX-master: t
%%% End:
